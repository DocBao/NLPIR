\chapter{特征噪声下的非贪婪度量学习}\label{chap:2}
\section{引言}
线性度量学习方法常旨在学习一个马氏度量矩阵$\boldsymbol{M}$刻画两个样本间的距离。由于$\boldsymbol{M}$矩阵是半正定的(positive semi-definite,PSD),所以一些方法常将$\boldsymbol{M}$分解为$\boldsymbol{W}\boldsymbol{W}^\mathrm{T}=\boldsymbol{M}$,转而求解一个投影矩阵刻画样本距离$d_{\boldsymbol{M}}(\boldsymbol{x}_i,\boldsymbol{x}_j)=\|\boldsymbol{W}^\mathrm{T}\boldsymbol{x}_i-\boldsymbol{W}^\mathrm{T}\boldsymbol{x}_j\|_2^2$。在$\boldsymbol{M}$非满秩的情况下,求解投影矩阵$\boldsymbol{W}$时的问题本质与降维(dimensionality reduction)相同 \citep{suvy_DML13,LDR_15jml}。因此很多诸如流形学习(manifold learning)一类的降维方法与度量学习有着较深的联系 \citep{dml_yang}。事实上,在高维空间中,学习一个刻画数据本征低维结构的度量,会使得算法的识别效果更好 \citep{Liao2015,pair17icml}。传统方法常采用L2距离损失作为样本损失,以刻画类内和类间散度,该类思想在样本观测分布准确无偏的假设前提下进行建模。但由于实际应用场景存在不可避免的噪声,L2损失加剧了噪声下的干扰程度,使得该类算法性能下降。

如何针对噪声环境构建一个鲁棒性的度量,是提升该情形下分类识别效果的关键 \citep{Yi2012}。纵观现有方法的出发点,传统方法本质上假设类内和类间样本服从两个不同高斯分布,由此该类方法的模型损失可等价为L2损失 \citep{DavidZhang}。为了降低噪声的干扰尺度、避免L2损失加剧观测分布的有偏程度,一部分工作转而采用L1损失进行建模 \citep{stocL1,niePCAL1}。该类方法本质是假设类内和类间样本服从两个不同的拉普拉斯分布 \citep{zhang2016exploring}。相对L2损失,L1损失的尺度较小,该类损失确实可以从一定程度上降低噪声的影响程度。因此,很多学者基于L1损失提出了一系列卓有成效的方法。例如,非监督情况下,PCA-L1 \citep{pca_L1}采用L1损失刻画总体样本方差,提升了算法的抗噪性。有监督情况下,基于L1损失的线性判别分析(linear discriminant analysis based on L1 loss, LDA-L1) \citep{zhong13}方法和Wang等人提出的度量学习方法 \citep{Wang2014a},均分别利用L1损失统一刻画类内散度和类间散度,推导了一种逐向量的贪婪求解算法,取得了一定的效果;为了弥补贪婪算法的缺陷,Liu等人 \citep{liu16non}在LDA-L1的基础上,推导了一种逐矩阵的非贪婪求解算法,提升了LDA-L1模型的性能。

然而,大部分上述方法存在两方面的问题。一方面,噪声使得观测距离偏大或偏小,对类内和类间样本优化方向具有不同的影响。统一采用L1损失只降低了距离尺度,并未分别考虑两类损失优化方向的差异性,不能从根本上提升度量的判别性。因此,考虑到这一点,扩大类内样本损失尺度、减小类间样本损失尺度,两者同时加大对两类样本的惩罚程度,可提升特征噪声下算法的判别性。与此同时,也有相关研究表明,类内样本和类间样本应考虑不同的损失 \citep{ye2018lp}。另一方面,现有采用L1损失的方法大部分采用逐向量的求解方式,未能实现关于矩阵变量的整体优化,求解效果不够理想。为了改善LDA-L1模型的求解问题,Liu等人 \citep{liu16non}推导了一种非贪婪的求解算法,但该方法是涉及到矩阵的梯度迭代,时间复杂度较高。

为此,基于边缘费歇尔分析方法 \citep{ShuichengYan2007,Xiong2014},本章提出了一种采用L2/L1损失的非贪婪的鲁棒性度量学习方法。该方法模型在最小化类内样本L2损失的同时最大化类间样本L1损失,比基于L2损失的方法赋予类间样本更大的惩罚,比基于L1损失的方法赋予类内样本更大的惩罚,该模型可同时提升度量的判别性。由于模型的目标函数非凸使得求解较为困难,本章随后将目标函数转为迭代优化两个凸函数之差;受凸函数差算法 (difference of convex functions algorithm, DCA) \citep{tao2005dc}思想启发,本章设计了一个辅助函数,由此推导了一种非贪婪的迭代求解算法;随后所给的理论证明保证了算法的收敛性。在公开数据集上,仿真实验对同类方法进行了不同噪声程度和噪声类型下的对比,结果表明了所提的MFA-L2/L1方法的优越性与鲁棒性。该工作的主要贡献如下:
\begin{itemize}
    \item 提出了一个基于L2/L1的边缘费歇尔分析模型,该模型利用L2损失刻画类内散度、L1损失刻画类间散度,解决噪声对类内和类间样本优化方向具有不同影响的问题,可同时提升度量的判别性;
    \item 设计并推导了一种非贪婪的迭代求解算法,在所构建的目标辅助函数下,解决了模型非凸的求解问题,该算法无需进行逐向量求解,具有较快的收敛性;
    \item 对迭代求解算法进行了理论分析,证明了其收敛性;
    \item 基于机器学习领域的公开数据集进行了特征噪声下的仿真实验,所提方法在对比实验中排名第一,结果验证了所提MFA-L2/L1方法的鲁棒性和有效性。
\end{itemize}

\section{相关工作}
假设给定训练样本集$\boldsymbol{X}=[\boldsymbol{x}_1,...,\boldsymbol{x}_n]$,其中,$\boldsymbol{x}_i\in \mathbb{R}^{d \times 1}$,表示第$i$个含有$d$维特征的训练样本。对任意样本对$(\boldsymbol{x}_i,\boldsymbol{x}_j)$,有$(\boldsymbol{x}_i,\boldsymbol{x}_j)\in \mathcal{S}$或$(\boldsymbol{x}_i,\boldsymbol{x}_j)\in \mathcal{D}$,其中,$\mathcal{S}$是正样本对集合(即相似样本对集合),$\mathcal{D}$是负样本对集合(即不相似样本对集合)。记待学习的投影矩阵为$\boldsymbol{W} \in \mathbb{R}^{d \times m}$,$m$表示$\boldsymbol{W}$其投影向量的个数。首先,这一节主要介绍与本章所提方法最相关的两个工作,基于L1损失的LDA-L1和基于L2损失的边缘费歇尔判别分析。

\subsection{基于L1损失的线性判别分析方法}
为了降低噪声对线性判别分析(linear discriminant analysis, LDA)的影响,一系列方法采用尺度较小的L1损失来改进LDA \citep{zhong13},这类方法致力于在优化目标函数下学习一个投影矩阵$\boldsymbol{W}$:
\begin{equation}\label{eq:L1LDA}
\max_{\boldsymbol{W},\boldsymbol{W}^\mathrm{T}\boldsymbol{W}=\boldsymbol{I}} \frac{\displaystyle \sum_{k=1}^c n_k\|(\overline{\boldsymbol{x}_k}-\boldsymbol{\overline{x}})^\mathrm{T}\boldsymbol{W} \|_1}{\displaystyle \sum_{k=1}^c \displaystyle \sum_{\boldsymbol{x}_{j} \in c_k}\|(\boldsymbol{x}_j-\overline{\boldsymbol{x}_k})^\mathrm{T}\boldsymbol{W} \|_1}.
\end{equation}
其中,$\overline{\boldsymbol{x}}$表示总体样本均值向量,$\overline{\boldsymbol{x}_k}$表示第$k$类样本均值向量;$n_k$表示第$k$类的样本数;$c$表示总类别数,$\boldsymbol{x}_{j} \in c_k$意味着样本$\boldsymbol{x}_j$属于第$k$类;$\boldsymbol{W}^\mathrm{T}\boldsymbol{W}=\boldsymbol{I}$保证了投影矩阵$\boldsymbol{W}$的正交性。不同于传统LDA,该类目标函数同时涉及L1范数的最大化和最小化,无法采用特征向量分解进行求解,使得模型求解较为困难。针对此类模型的求解问题,LDA-L1采用逐向量优化的贪婪求解方法,该优化过程性能相对逐矩阵的优化算法略差 \citep{liu16non}。针对这一问题,Liu等人近来提出了一种针对L1范数LDA的非贪婪算法(a non-greedy algorithm for L1-norm LDA, LDA-NgL1) \citep{liu16non},针对同一模型~\eqref{eq:L1LDA},该算法对矩阵进行整体求解,提升了模型的性能。但该方法涉及矩阵的梯度迭代,计算复杂度较高。

\subsection{边缘费歇尔分析方法}
由于真实应用较为复杂,数据分布不一定服从LDA关于高斯分布\citep{ShuichengYan2007}的假设。为了克服这一局限性,样本无需满足高斯分布,边缘费歇尔分析(marginal fisher analysis, MFA) \citep{ShuichengYan2007}根据近邻样本的相似性重新定义了类内散度与类间散度,该方法取得了一定的成功和较为广泛的应用\citep{mfaAp1,mfaAp2}。基于样本近邻关系,该方法采用L2距离作为损失来定义类内散度和类间散度,在优化目标下学习投影矩阵$\boldsymbol{W}$:
\begin{equation}\label{eq:MFA}
\min_{\boldsymbol{W},\boldsymbol{W}^\mathrm{T}\boldsymbol{W}=\boldsymbol{I}} \frac{\displaystyle \sum_{\boldsymbol{x}_{ij}\in \mathcal{S_M}}\|\boldsymbol{x}_{ij}^\mathrm{T}\boldsymbol{W}\|_2^{2}}{\displaystyle \sum_{\boldsymbol{x}_{ij}\in \mathcal{D_M}}\|\boldsymbol{x}_{ij}^\mathrm{T}\boldsymbol{W}\|_2^{2}}.
\end{equation}
此处,为便于理解不同模型之间的差异性,区别于MFA原始模型的矩阵表达方式,模型~\eqref{eq:MFA}采用样本对形式表示目标函数。其中,$\boldsymbol{x}_{ij}=\boldsymbol{x}_i-\boldsymbol{x}_j$是正样本对或负样本对之差,行向量$\boldsymbol{x}_{ij}^\mathrm{T}$是关于列向量$\boldsymbol{x}_{ij}$的转置。$\mathcal{S_M}$是图$G_{\mathcal{S}}=\{\boldsymbol{X},\boldsymbol{A}_{\mathcal{S}}\}$所连接的相似样本对集,而图$G_{\mathcal{S}}$是刻画类内散度的本质图(intrinsic graph),即$\mathcal{S_M}$表示正样本对集;同理,$\mathcal{D_M}$是图$G_{\mathcal{D}}=\{\boldsymbol{X},\boldsymbol{A}_{\mathcal{D}}\}$所连接的不相似样本对集,而图$G_{\mathcal{D}}$是刻画类间散度的惩罚图(penalty graph),即$\mathcal{D_M}$表示负样本对集。对于两个图的连接矩阵$\boldsymbol{A}_{\mathcal{S}}$和$\boldsymbol{A}_{\mathcal{D}}$,其定义分别如下:

\begin{equation} \label{eq:MFA_Ls}
  \boldsymbol{A}_{\mathcal{S}}(i,j)=\left\{
   \begin{aligned}
   &1,&&{\text{若}i \in N^{+}_{k_{\mathcal{S}}}(j)\text{或} j \in N^{+}_{k_{\mathcal{S}}}(i)};\\
   &0,&&{\text{其它}}.
   \end{aligned}
   \right.\\
\end{equation}
\begin{equation} \label{eq:MFA_Lb}
   \boldsymbol{A}_{\mathcal{D}}(i,j)=\left\{
   \begin{aligned}
   &1,&&{\text{若}i \in N^{-}_{k_{\mathcal{D}}}(j) \text{或}j \in N^{-}_{k_{\mathcal{D}}}(i)};\\
   &0,&&{\text{其它}}.
   \end{aligned}
   \right.\\
\end{equation}
其中,$N^{+}_{k_{\mathcal{S}}}(i)$表示与第$i$个样本具有相同标签的$k_{\mathcal{S}}$个近邻的样本集合,$N^{-}_{k_{\mathcal{D}}}(i)$表示与其具有不同标签的$k_{\mathcal{D}}$个近邻的样本集合。该目标函数可直接通过特征向量分解实现求解。但由于MFA模型采用了L2距离作为损失,其性能易受噪声的影响。

\section{L2/L1损失的边缘费歇尔分析算法}
基于MFA方法,针对噪声下观测分布的有偏性,本小节基于对噪声的影响分析提出特征噪声下的度量学习模型,对于模型求解的困难性提出了一种非贪婪的求解算法,并对算法收敛性进行了证明分析。
\subsection{特征噪声下的度量学习模型}
由于特征噪声下样本的观测分布偏离真实分布,因此在该情形下样本的观测距离相对真实情况可能会偏大或偏小,使得噪声对类内和类间样本优化方向具有不同的影响。结合图~\ref{fig:NsyPoint}进行具体分析,对类内散度而言,偏小的观测距离可能使得模型对正样本对的惩罚程度不足。因此,相对L1损失,采用尺度较大的L2损失刻画类内散度,加大对正样本对的惩罚程度更为合适。相反,对类间散度而言,观测距离偏大可能使得模型对负样本对的惩罚程度不足,采用尺度较小的L1损失刻画类间散度,加大对负样本对的惩罚程度更为合适。
  
\begin{figure}[!htbp]
    \centering
    \includegraphics[width=0.8\textwidth]{chap2_NsyPoint}
    \caption{特征噪声下的观测分布与真实分布对比图}
    \label{fig:NsyPoint}
\end{figure}

考虑MFA方法无需数据满足高斯分布的前提优势,本章提出一种基于L2/L1损失的边缘费歇尔分析(marginal fisher analysis based on L2/L1 loss, MFA-L2/L1)模型:

\begin{equation}\label{eq:MFA21}
\begin{aligned}
\min_{\boldsymbol{W},\boldsymbol{W}^\mathrm{T}\boldsymbol{W}=\boldsymbol{I}} &\frac{\displaystyle \sum_{\boldsymbol{x}_{ij}\in \mathcal{S_M}}\|\boldsymbol{x}_{ij}^\mathrm{T}\boldsymbol{W}\|_{2}^{2}}{\displaystyle \sum_{\boldsymbol{x}_{ij}\in \mathcal{D_M}}\|\boldsymbol{x}_{ij}^\mathrm{T}\boldsymbol{W}\|_1}=\min_{\boldsymbol{W},\boldsymbol{W}^\mathrm{T}\boldsymbol{W}=\boldsymbol{I}}\frac{tr(\boldsymbol{W}^\mathrm{T}\boldsymbol{A}^\mathrm{T}\boldsymbol{A}\boldsymbol{W})}{\|\boldsymbol{BW}\|_1}.
\end{aligned}
\end{equation}
与MFA的目标函数~\eqref{eq:MFA}类似,$\boldsymbol{x}_{ij}$表示基于近邻关系的正样本对集$\mathcal{S_M}$或负样本对集$\mathcal{D_M}$的样本差向量;$tr(\cdot)$表示对矩阵取迹的算子;矩阵$\boldsymbol{A}$的行向量为所有来自正样本对集$\mathcal{S_M}$的行向量$\boldsymbol{x}_{ij}^\mathrm{T}$;矩阵$\boldsymbol{B}$的行向量为所有来自负样本对集$\mathcal{D_M}$的行向量$\boldsymbol{x}_{ij}^\mathrm{T}$;需要说明和区分的是,为了表示方便,本章参考文献 \citep{stocL1}的表示方式,对任意大小为$n_1 \times n_2$矩阵$\boldsymbol{C}$,定义其范数$\footnote{本章中该形式矩阵范数均只采用这一种定义,其形式虽与矩阵的列和范数形式相同,但计算方式并不相同,请注意区分。}$ $\|\boldsymbol{C}\|_1=\sum_{i=1}^{n_1} \sum_{j=1}^{n_2}|C_{ij}|$。

与LDA-L1、MFA两个方法对比,MFA-L2/L1方法的目标函数不仅非凸,且其同时涉及L2范数最小化和L1范数最大化。因此,LDA-L1、LDA-NgL1和MFA的求解算法都不将再适合本章所提的MFA-L2/L1模型。为了解决该模型的求解问题,下一小节将作进一步分析。

\subsection{非贪婪的求解算法}
分析MFA-L2/L1模型,其目标函数是关于两个凸函数的商形式,且这两个刻画散度的凸函数值始终非零。因此,受文献 \citep{wang2014globally,Wang2014a}中的定理1和定理2启发,为使得目标函数~\eqref{eq:MFA21}下降,本章设计了一个关于$\boldsymbol{W}_t$和$\lambda_{t-1}$的求解形式,转为迭代两个凸函数之差:
\begin{equation}\label{eq:RMFA}
\boldsymbol{W}_t=\arg \min_{\boldsymbol{W},\boldsymbol{W}^\mathrm{T}\boldsymbol{W}=\boldsymbol{I}}G(\boldsymbol{W})- \lambda_{t-1}L(\boldsymbol{W}).
\end{equation}
在公式~\eqref{eq:RMFA}中,$G(\boldsymbol{W})= tr(\boldsymbol{W}^\mathrm{T}\boldsymbol{A}^\mathrm{T}\boldsymbol{AW})$,$L(\boldsymbol{W})=\|\boldsymbol{BW}\|_1$,$\lambda_{t-1}$是第$t-1$次迭代时关于投影矩阵$\boldsymbol{W}_{t-1}$的目标函数值:
\begin{equation}\label{eq:RMFA_itr1}
\lambda_{t-1} = \frac{tr(\boldsymbol{W}_{t-1}^\mathrm{T}\boldsymbol{A}^\mathrm{T}\boldsymbol{A}\boldsymbol{W}_{t-1})}{\|\boldsymbol{B}\boldsymbol{W}_{t-1}\|_1}.
\end{equation}
为了更为简洁的表达,用函数$F(\boldsymbol{W})$表示$G(\boldsymbol{W})- \lambda_{t-1}L(\boldsymbol{W})$,则公式~\eqref{eq:RMFA}的目标函数可记为:
\begin{equation}\label{eq:RMFA_itr2}
\boldsymbol{W}_t=\arg \min_{\boldsymbol{W},\boldsymbol{W}^\mathrm{T}\boldsymbol{W}=\boldsymbol{I}}F(\boldsymbol{W}).
\end{equation}

分析公式~\eqref{eq:RMFA_itr2},目标函数是两个凸函数$G(\boldsymbol{W})$和$\lambda_{t-1}L(\boldsymbol{W})$之差,$\lambda_{t-1}$在公式中是常数,不影响第$t$次迭代时的推导分析。由于公式~\eqref{eq:RMFA_itr2}的目标函数具有与DCA思想 \citep{tao2005dc}相同的差形式,受该思想启发,公式~\eqref{eq:RMFA_itr2}在第$t$次迭代求解$\boldsymbol{W}_t$时,可生成两个变量序列,即中间变量序列${\boldsymbol{Y}^k}$和求解变量序列${\boldsymbol{W}^k}$(为了与外部$\lambda_{t-1}$的交替迭代次序区分开来,上标$k$表示在固定第$t$次迭代时采用DCA思想求解$\boldsymbol{W}_t$的第$k$次内部迭代):
\begin{equation}\label{eq:RMFA_SEQ}
\begin{split}
\left\{
   \begin{aligned}
   &\boldsymbol{Y}^k\in \partial L(\boldsymbol{W}^k),\\
   &\boldsymbol{W}^{k+1}=\arg \min_{\boldsymbol{W},\boldsymbol{W}^\mathrm{T}\boldsymbol{W}=\boldsymbol{I}} G(\boldsymbol{W})-\lambda_{t-1} \left[  L(\boldsymbol{W}^k)+tr\left( \left( \boldsymbol{Y}^k \right) ^\mathrm{T} \left(\boldsymbol{W}-\boldsymbol{W}^k \right)\right) \right] .
   \end{aligned}
\right.\\
\end{split}
\end{equation}

%$\boldsymbol{Y}^k \odot (\boldsymbol{W}-\boldsymbol{W}^k)$表示$\boldsymbol{Y}^k$与$\boldsymbol{W}-\boldsymbol{W}^k$的Hadamard积。由此可知,$L(\boldsymbol{W}^k)+\|\boldsymbol{Y}^k \odot (\boldsymbol{W}-\boldsymbol{W}^k)\|_1$

其中,$\boldsymbol{Y}^k$是$L(\boldsymbol{W}^k)$在$\boldsymbol{W}^k$处的次梯度;则$L(\boldsymbol{W}^k)+tr\left( \left( \boldsymbol{Y}^k \right) ^\mathrm{T} \left(\boldsymbol{W}-\boldsymbol{W}^k \right)\right)$可看作是$L(\boldsymbol{W})$在$\boldsymbol{W}^k$处的一阶展开式。记$H(\boldsymbol{W},\boldsymbol{W}^k)=tr\left( \left( \boldsymbol{Y}^k \right) ^\mathrm{T} \left(\boldsymbol{W}-\boldsymbol{W}^k \right)\right)$,由此,对于序列~\eqref{eq:RMFA_SEQ}有如下定理:

\begin{theorem}
\label{theory:theory_1}
   序列~\eqref{eq:RMFA_SEQ}中的$\boldsymbol{W}^{k+1} \in \mathbb{R}^{d \times m}$使得$F(\boldsymbol{W}^{k+1})\leq F(\boldsymbol{W}^k)$成立。 
\end{theorem}

\begin{proof}
由于$\boldsymbol{Y}^k$是$L(\boldsymbol{W})$在$\boldsymbol{W}^k$处的次梯度,所以在$\boldsymbol{W}^k$邻域内对任意$\boldsymbol{W}$有:
$ L(\boldsymbol{W}^k)+H(\boldsymbol{W},\boldsymbol{W}^k) \leq L(\boldsymbol{W}) $成立。则:

\begin{equation}\label{ineq::INEQ12}
\begin{aligned}
F(\boldsymbol{W}^{k+1})&=G(\boldsymbol{W}^{k+1})-\lambda_{t-1} L(\boldsymbol{W}^{k+1})\notag\\
&\leq G(\boldsymbol{W}^{k+1})-\lambda_{t-1} [ L(\boldsymbol{W}^k)+H(\boldsymbol{W^{k+1}},\boldsymbol{W}^k)]\notag\\
&\leq G(\boldsymbol{W}^{k})-\lambda_{t-1} [ L(\boldsymbol{W}^k)+H(\boldsymbol{W^k},\boldsymbol{W}^k)]\notag\\
&=G(\boldsymbol{W}^k)-\lambda_{t-1} L(\boldsymbol{W}^k)=F(\boldsymbol{W}^k).
\end{aligned}
\end{equation}
\end{proof}

根据定理~\eqref{theory:theory_1}可知,由序列~\eqref{eq:RMFA_SEQ}可得到关于公式~\eqref{eq:RMFA_itr2}中目标函数的下降序列$F(\boldsymbol{W}^k)$,因而由序列~\eqref{eq:RMFA_SEQ}可推导出具体迭代为:

\begin{numcases}{}
%\left\{
\boldsymbol{Y}^k = \boldsymbol{B}^\mathrm{T}\emph{sign}(\boldsymbol{B}\boldsymbol{W}^{k}),\label{eq:SqObj1}\\
\boldsymbol{W}^{k+1}=\arg \min_{\boldsymbol{W},\boldsymbol{W}^\mathrm{T}\boldsymbol{W}=\boldsymbol{I}} tr(\boldsymbol{W}^\mathrm{T}\boldsymbol{A}^\mathrm{T}\boldsymbol{A}\boldsymbol{W})-\lambda_{t-1} H(\boldsymbol{W},\boldsymbol{W}^k)\label{eq:SqObj2}.
%\right.\\
\end{numcases}
其中,$\emph{sign}(\cdot)$是关于矩阵逐元素的函数:
\begin{equation}
\begin{split}
\emph{sign}(x)=\left\{
   \begin{aligned}
1, &\text{若} \quad x  \geq 0,\\
-1, &\text{其它}.
   \end{aligned}
\right.\\
\end{split}
\end{equation}
值得注意的是,求解$\boldsymbol{W}^{k+1}$时,在序列~\eqref{eq:SqObj2}中关于$\boldsymbol{W}^{k}$的常数项部分$\lambda_{t-1} L(\boldsymbol{W}^k)$已忽略。将序列~\eqref{eq:SqObj2}的目标函数关于矩阵$\boldsymbol{W}$求导,并令导数为零矩阵,得到:
\begin{equation}\label{eq:GradW}
2\boldsymbol{A}^\mathrm{T}\boldsymbol{A}\boldsymbol{W}-\lambda_{t-1}\boldsymbol{Y}^k=\boldsymbol{0}.
\end{equation}

解之得:
\begin{equation}\label{eq:SoluW}
\boldsymbol{W}^{k+1}=\frac{\lambda_{t-1}}{2}(\boldsymbol{A}^\mathrm{T}\boldsymbol{A})^{-1}\boldsymbol{Y}^k.
\end{equation}

按关于$\boldsymbol{Y}^k$的序列~\eqref{eq:SqObj1}和关于$\boldsymbol{W}^{k+1}$的迭代公式~\eqref{eq:SoluW}进行计算,直至序列~\eqref{eq:SqObj2}迭代停止。该迭代过程所得到的$\boldsymbol{W}^{k+1}$就是第$t$次迭代所求的$\boldsymbol{W}_t$,进一步可由公式~\eqref{eq:RMFA_itr1}的原理计算$\lambda_{t}$。

为保证矩阵$\boldsymbol{W}$的正交性约束,该问题可通过正交强迫一致问题(orthogonal procrustes problem) \citep{ocp1952}来解决,即:$\boldsymbol{W}=\arg \min_{\boldsymbol{W},\boldsymbol{W}^\mathrm{T}\boldsymbol{W}=\boldsymbol{I}}\|\boldsymbol{W}-\boldsymbol{W}_t\|_{F}$。该问题的解为$\boldsymbol{W}=\boldsymbol{U}\boldsymbol{V}^\mathrm{T}$,其中,$\boldsymbol{U}$和$\boldsymbol{V}^\mathrm{T}$分别是来自$\boldsymbol{W}_t$奇异值分解(Singular Value Decomposition, SVD)中大小为$d \times k_1$和$k_1 \times m$的正交矩阵:$\boldsymbol{U\Sigma }\boldsymbol{V}^\mathrm{T}=\boldsymbol{W}_t$,其对应的非零奇异值个数为$k_1$。总结以上推导过程,MFA-L2/L1模型的具体求解步骤如算法~\ref{alg:MFA-L2/L1}所示。

\begin{algorithm}[!htbp]

\renewcommand\baselinestretch{1.2}\selectfont
\caption{MFA-L2/L1求解算法}\label{alg:MFA-L2/L1}
\begin{algorithmic}[1]
\State \textbf{输入:} {$\boldsymbol{X}=[\boldsymbol{x}_1,...,\boldsymbol{x}_n]$, $\boldsymbol{x}_i\in \mathbf{R}^d$,总内部迭代次数$K$,外部总迭代次数$T$,投影矩阵向量数$m$;}
\State \textbf{初始化$\boldsymbol{W}_0$}.
\While{迭代终止条件不满足}
\State 更新 $\lambda_{t-1} = \frac{tr(\boldsymbol{W}_{t-1}^\mathrm{T}\boldsymbol{A}^\mathrm{T}\boldsymbol{A}\boldsymbol{W}_{t-1})}{\|\boldsymbol{B}\boldsymbol{W}_{t-1}\|_1}$;
\State $k=1$;
\While{序列~\eqref{eq:SqObj2}不收敛}
\State 计算$\boldsymbol{Y}^k = \boldsymbol{B}^\mathrm{T}\emph{sign}(\boldsymbol{B}\boldsymbol{W}^{k})$;
\State 计算$\boldsymbol{W}^{k+1}=\frac{\lambda_{t-1}}{2}(\boldsymbol{A}^\mathrm{T}\boldsymbol{A})^{-1}\boldsymbol{Y}^k$;
\State $k=k+1$;
\EndWhile %\label{MFA-L2/L1Eendwhile}
\State 计算SVD分解: $\boldsymbol{U\Sigma }\boldsymbol{V}^\mathrm{T}=\boldsymbol{W}_t$;
\State 更新$\boldsymbol{W}_t=\boldsymbol{U}\boldsymbol{V}^\mathrm{T}$;
\State $t=t+1$.
\EndWhile  %\label{MFA-L2/L1Eendwhile}
\State \textbf{输出:}投影矩阵$\boldsymbol{W}$.
\end{algorithmic}
\end{algorithm}

分析迭代流程,序列$\boldsymbol{Y}^k$的作用是构造一个关于公式~\eqref{eq:RMFA_itr2}中目标函数的辅助函数,即序列~\eqref{eq:SqObj2}中关于$\boldsymbol{W}^k$的目标函数,以确保迭代过程中公式~\eqref{eq:SqObj2}的目标函数下降。整个算法始终对$\boldsymbol{W}$进行整体求解,属于非贪婪算法。由于以上分析仅限于在固定$\lambda_{t-1}$、求解$\boldsymbol{W}_t$时目标函数~\eqref{eq:RMFA_itr2}的单调性情况,下一小节将联合变量$\boldsymbol{W}_t$和$\lambda_t$分析算法的整体收敛性。

\subsection{算法收敛性及计算开销分析}
\begin{theorem}
\label{theory:theory_2}
   算法~\ref{alg:MFA-L2/L1}使得目标函数~\eqref{eq:MFA21}下降。 
\end{theorem}
\begin{proof}
当固定第$t$次外部迭代时,关于$\boldsymbol{W}$的梯度为零时有公式~\eqref{eq:GradW}成立;此外,由定理~\ref{theory:theory_1}可知,$\boldsymbol{W}_t$使得目标函数$F\left(\boldsymbol{W}\right)$下降,将$\boldsymbol{W}_t$拆分成关于$\left(\boldsymbol{w}_l\right)_t$投影向量形式的表达,用$\boldsymbol{b}^i$表示矩阵$\boldsymbol{B}$的第$i$个行向量,则成立:
  \begin{align}\label{eq:covg1}
    &\displaystyle \sum_{l=1}^m \left(\boldsymbol{w}_l\right)_{t}^\mathrm{T}\boldsymbol{A}^\mathrm{T}\boldsymbol{A}\left(\boldsymbol{w}_l\right)_{t}-\lambda_{t-1}\displaystyle \sum_{l=1}^m \sum_{i} \frac{\left(\boldsymbol{w}_l\right)_t^\mathrm{T}{\left( \boldsymbol{b}^i\right) }^\mathrm{T}\boldsymbol{b^i}\left(\boldsymbol{w}_l\right)_{t-1}}{|\boldsymbol{b}^i\left(\boldsymbol{w}_l\right)_{t-1}|}\leq \notag\\
    &\displaystyle \sum_{l=1}^m \left(\boldsymbol{w}_l\right)_{t-1}^\mathrm{T}\boldsymbol{A}^\mathrm{T}\boldsymbol{A} \left(\boldsymbol{w}_l \right)_{t-1}-\lambda_{t-1}\displaystyle \sum_{l=1}^m\sum_{i}  \frac{\left(\boldsymbol{w}_l\right)^\mathrm{T}_{t-1}{\left( \boldsymbol{b}^i\right) }^\mathrm{T}\boldsymbol{b}^i\left(\boldsymbol{w}_l\right)_{t-1}}{|\boldsymbol{b}^i\left(\boldsymbol{w}_l\right)_{t-1}|}.
    \end{align}
对于矩阵$\boldsymbol{B}$的第$i$个行向量$\boldsymbol{b}^i$,根据柯西-施瓦兹不等式,在第$t$次迭代与第$t-1$次迭代之间,对投影矩阵$\boldsymbol{W}$的第$l$个投影列向量成立:
  \begin{equation}\label{eq:covg2}
    \left(\boldsymbol{b}^i\left(\boldsymbol{w}_l\right)_t\right)^\mathrm{T}\boldsymbol{b}^i\left(\boldsymbol{w}_l\right)_{t-1}\leq |\boldsymbol{b}^i\left(\boldsymbol{w}_l\right)_t||\boldsymbol{b}^i\left(\boldsymbol{w}_l\right)_{t-1}|.
  \end{equation}  
将矩阵$\boldsymbol{B}$的所有行向量关于以上形式的不等式变形,并进行叠加,则成立:
  \begin{align}\label{eq:covg3}
    &\displaystyle \sum_{i} \frac{\left(\boldsymbol{b}^i\left(\boldsymbol{w}_l\right)_t\right)^\mathrm{T}\boldsymbol{b}^i\left(\boldsymbol{w}_l\right)_{t-1}}{|\boldsymbol{b}^i\left(\boldsymbol{w}_l \right)_{t-1}|}-\displaystyle \sum_{i} |\boldsymbol{b}^i\left(\boldsymbol{w}_l\right)_{t}|\leq \notag\\
    &\displaystyle \sum_{i} \frac{\left(\boldsymbol{b}^i\left(\boldsymbol{w}_l\right)_{t-1}\right)^\mathrm{T}\boldsymbol{b}^i\left(\boldsymbol{w}_l\right)_{t-1}}{|\boldsymbol{b}^i\left(\boldsymbol{w}_l\right)_{t-1}|}-\displaystyle \sum_{i} |\boldsymbol{b}^i\left(\boldsymbol{w}_l\right)_{t-1}|=0.
  \end{align}
将不等式~\eqref{eq:covg3}带入不等式~\eqref{eq:covg1},得到:
  \begin{align}\label{eq:covg4}
    &\displaystyle \sum_{l=1}^m \left(\boldsymbol{w}_l\right)_{t}^\mathrm{T}\boldsymbol{A}^\mathrm{T}\boldsymbol{A}\left(\boldsymbol{w}_l\right)_{t}-\lambda_{t-1}\displaystyle \sum_{l=1}^m \| B\left(\boldsymbol{w}_l\right)_{t}\|_1\leq \notag\\
    &\displaystyle \sum_{l=1}^m \left(\boldsymbol{w}_l\right)_{t-1}^\mathrm{T}\boldsymbol{A}^\mathrm{T}\boldsymbol{A}\left(\boldsymbol{w}_l\right)_{t-1}-\lambda_{t-1}\displaystyle \sum_{l=1}^m \| B\left(\boldsymbol{w}_l\right)_{t-1}\|_1=0.
  \end{align}
最后成立:
 \begin{equation}\label{eq:covg5}
    \frac{\displaystyle \sum_{l=1}^m \|\boldsymbol{A}\left(\boldsymbol{w}_l\right)_{t}\|_{2}^{2}}{\displaystyle \sum_{l=1}^m \|\boldsymbol{B}\left(\boldsymbol{w}_l\right)_{t}\|_1}\leq \lambda_{t-1}=\frac{\displaystyle \sum_{l=1}^m \|\boldsymbol{A}\left(\boldsymbol{w}_l\right)_{t-1}\|_{2}^{2}}{\displaystyle \sum_{l=1}^m \|\boldsymbol{B}\left(\boldsymbol{w}_l\right)_{t-1}\|_1}.
  \end{equation}
即目标~\eqref{eq:MFA21}下降。
\end{proof}

通过算法1可知,该算法的时间复杂度主要来自于外部循环步骤4、11、12与内部循环步骤7、8这五个步骤。若矩阵$\boldsymbol{A}$的类内样本对差向量个数为$n_1$,矩阵$\boldsymbol{B}$的类内样本对差向量个数为$n_2$,SVD分解中非零奇异值个数为$k_1$,则五个步骤的计算开销与外部迭代次数$T$、内部迭代次数$K$、投影向量个数$m$和空间维数$d$有关。分析可知步骤4、7、8、11、12的时间复杂度分别为$O\left((d+1)n_1m+(d+1)n_2m\right)$、$O\left(2n_2md\right)$、$O\left((n_1+m+d)d^2\right)$、$O\left(d^3\right)$、$O\left(k_1md\right)$。因此,$T \cdot O\left(\left[ k_1d+(n_1+n_2)(d+1)\right] m+d^3+ K  \cdot \left[ 2n_2m+(n_1+m+d)d \right]d  \right)$为该算法的时间复杂度,$T \cdot O\left(\left[ k_1d+(n_1+n_2)(d+1)\right] m+d^3+ \left[ 2n_2m+(n_1+m+d)d \right] d  \right)$是内部循环次数$K=1$时所对应的时间复杂度。根据实验观察,表明内部迭代仅需几次即可终止。因此,算法的时间复杂度可参考内部循环次数$K=1$时所对应的时间复杂度。

 
\section{实验分析}
为了验证MFA-L2/L1方法的鲁棒性和解法的有效性,实验在5个UCI数据集和7个人脸数据集上添加了不同程度的特征噪声,与相关方法进行了对比。为保证实验的公正性,所有实验均从训练样本集中学习一个度量矩阵或投影矩阵,基于所学度量采用最近邻算法对测试样本进行分类。

\subsection{UCI数据集实验}
为分析所提算法在不同噪声程度下的性能,实验分别在UCI数据集$\footnote{http://archive.ics.uci.edu/ml/datasets.php}$中选用Statlog (Heart)、Parkinsons \citep{Parkinsons2007}、Seeds、Sonar、Wine加入了比例为$5\%$、$15\%$、$25\%$、$30\%$左右的特征噪声,进行同类方法对比。以上UCI数据集信息见表~\ref{tab:chap2_UCIdata}。为避免特征尺度与噪声尺度不一致而影响评价的公正性,实验将数据集每个特征向量的模标准化为1,参考椒盐噪声的方式,实验按噪声比例的平方根进行样本数和特征数取整,在数据矩阵中随机选取对应样本数下对应特征数的矩阵元素,将其置为0或1作特征噪声。每个数据集按7:3的比例随机划分为训练集和测试集,该随机过程重复100次,报告平均准确率及方差作为实验结果。
\begin{table}[htbp]
{\renewcommand\baselinestretch{0.84}\selectfont
\renewcommand{\arraystretch}{1.3}
\caption{所使用UCI数据信息表}
\label{tab:chap2_UCIdata}
\centering
\scalebox{1}{
\begin{threeparttable}
\begin{tabular}{|l|c|c|c|}
\hline
数据集 &{样本数 (n)}  &{特征数 (d)} &{类别数 (c)} \\  \hline
\hline
Statlog (Heart) &270 &13 &2\\
Parkinsons &195 &22 &2\\
Seeds &210 &7 &3\\
Sonar &208 &60 &2\\
Wine &178 &13 &3\\
\hline
\end{tabular}
\scriptsize
\end{threeparttable}}
\par}
\end{table}

实验选用LDA-L1 \citep{zhong13}、LDA-NgL1 \citep{liu16non}、MFA-NgL1、MFA\citep{ShuichengYan2007}、discriminative null space(DNS) \citep{DNS16}作为对比方法,与MFA-L2/L1进行比较。需要指出说明的是,MFA-NgL1是在MFA的样本关系下,采用LDA-NgL1相同的L1损失与求解算法。这些方法中,LDA-L1、LDA-NgL1、MFA-NgL1和MFA-L2/L1均属于迭代方法;MFA和DNS属于非迭代方法。LDA-L1的梯度步长设为0.01,LDA-NgL1和MFA-NgL1设置均遵循LDA-NgL1 \citep{liu16non}原代码。经实验测试,固定矩阵$\boldsymbol{W}$的第$t$次迭代,MFA-L2/L1序列内部迭代仅需一次即可收敛,因此最大次数设置为$K=1$;LDA-NgL1和MFA-NgL1采用的解法属于矩阵梯度迭代法,需要较高的迭代次数,因此第$t$次迭代时将内部迭代次数$T$设置为100;当这三个迭代方法的迭代次数满足$t \ge 100$或迭代误差满足以下条件时,算法终止:
\begin{equation}\label{eq:iter_err}
    \delta=\frac{\|\boldsymbol{W}_{t+1}-\boldsymbol{W}_t\|_F}{\|\boldsymbol{W}_t\|_F} \le e^{-11}.
\end{equation}

对于需要矩阵初始化的方法,均在同一随机种子下生成同一初始矩阵,该过程基于每次划分训练集和测试集时的不同随机种子生成。实验事先采用PCA计算$95\%$贡献率的成分数$p$,设置投影矩阵输出维度为$p+1$。最终结果见图~\ref{fig:chap2_UCI},算法排名结果见表~\ref{tab:chap2_UCIRk}。

根据图~\ref{fig:chap2_UCI}和表~\ref{tab:chap2_UCIRk}的的报告结果可知,在不同的噪声程度下,MFA-L2/L1相对其它同类方法更优,具有较好的准确性和鲁棒性。在Parkinson、Seeds和Wine三个数据集上,MFA-L2/L1在四种噪声程度下的性能均排名第一。观察Wine数据集上的结果,所提方法与次优方法LDA-NgL1的差异尤为明显,在$15\%$至$30\%$三个噪声程度下性能两者性能差异超过了$9\%$。在图~\ref{fig:chap2_UCI1}中,所提方法虽然在Heart和Sonar数据集上性能略低于LDA-NgL1、MFA-NgL1和LDA-L1,但观察图~\ref{fig:chap2_UCI2}至图~\ref{fig:chap2_UCI4}可发现:随噪声程度增大,MFA-L2/L1在这两个数据集上逐渐胜过其它方法,其对于次优方法的优势逐渐增大。因此,基于本次实验中不同数据集、不同噪声程度的对比结果说明,随噪声程度增加,MFA-L2/L1表现了较好的抗噪性和识别性。

\begin{figure}[!htbp]
%\subfigure[1]{
\centering
    \begin{subfigure}[b]{0.49\textwidth}
      \includegraphics[width=\textwidth]{chap2_UCI_1.eps}
      \caption{}
      \label{fig:chap2_UCI1}
    \end{subfigure}%
    ~% add desired spacing
    \begin{subfigure}[b]{0.49\textwidth}
      \includegraphics[width=\textwidth]{chap2_UCI_2.eps}
      \caption{}
      \label{fig:chap2_UCI2}
    \end{subfigure}%
    \\ \vspace{2mm}%% line break

    \begin{subfigure}[b]{0.49\textwidth}
      \includegraphics[width=\textwidth]{chap2_UCI_3.eps}
      \caption{}
      \label{fig:chap2_UCI3}
    \end{subfigure}%
    ~% add desired spacing
    \begin{subfigure}[b]{0.49\textwidth}
      \includegraphics[width=\textwidth]{chap2_UCI_4.eps}
      \caption{}
      \label{fig:chap2_UCI4}
    \end{subfigure}%
    %\\% line break
    \caption{UCI数据集上MFA-L2/L1与各方法的对比图 (a) $5\%$噪声程度 (b) $15\%$噪声程度 (c) $25\%$噪声程度 (d) $30\%$噪声程度}
    \label{fig:chap2_UCI}
\end{figure}

\begin{table}[htbp]
\small
{\renewcommand\baselinestretch{0.9}\selectfont
\renewcommand{\arraystretch}{1.3}
    \caption{各算法基于UCI数据集的实验结果排名}
    \label{tab:chap2_UCIRk}
\centering
\scalebox{1}{
\begin{threeparttable}
		\begin{tabular}{|l|c|c|c|c|c|c|}\hline 
算法 &{MFA-L2/L1}  &{LDA-NgL1} &{MFA-NgL1} &{LDA-L1} &{MFA} &{DNS}\\   \hline
\hline
排名 &{\textbf{1.4}}  &{2.75} &{3.15} &{4.6} &{3.7} &{5.4}\\ 
\hline
\end{tabular}
\footnotesize
  最优结果加粗表示。
\end{threeparttable}}
\par}
\end{table}

\subsection{AR数据集实验} 

鉴于MFA-L2/L1和MFA两者之间的联系,为了更全面地比较这两个方法的性能,该小节在AR数据集上进行实验对比。AR数据集 \citep{AR01}是人脸识别领域较为经典的数据集之一,选取该数据集裁剪后的2600张人脸图像进行实验,共有100人(男性、女性各50人)。实验中,每个人的图像属于一类,每类共含26张图像,具有光照、表情、是否佩戴眼镜或围巾等变化。

实验事先将所有图像转为50×40像素大小的灰度图像,随机从每个类别中选取6张图像,分别加入$15\%$比例的高斯块和椒盐噪声两个类型噪声。随后从每个类别中随机选取13张图像作为训练样本,剩余13张图像作为测试样本。为避免实验结果的随机性,该过程重复10次后取平均结果进行报告。所有样本采用PCA保留$95\%$贡献率作为前处理,其余实验设置与上一小节关于UCI数据集的实验设置相同。实验对比了投影矩阵维度在80维至225维之间的性能变化,将其结果报告如图~\ref{fig:AR}。
\begin{figure}[!htbp]
\centering
    \begin{subfigure}[b]{0.7\textwidth}
      \includegraphics[width=\textwidth]{chap2_AR_1}
      \caption{}
      \label{fig:AR_1}
    \end{subfigure}%
    \\ %\hspace{10mm}% line break
 %\hspace{10mm}% line break
    \begin{subfigure}[b]{0.7\textwidth}
      \includegraphics[width=\textwidth]{chap2_AR_2}
      \caption{}
      \label{fig:AR_2}
    \end{subfigure}%
   % \\% line break
    \caption{MFA-L2/L1与MFA-L2两个方法基于AR数据随维度变化的准确率对比图(\%) (a) 噪声类型-高斯块噪声 (b) 噪声类型-椒盐噪声}
    \label{fig:AR}
\end{figure}

观察图~\ref{fig:AR}可知,在高斯块噪声情况下,MFA-L2/L1性能整体优于MFA。在椒盐噪声情况下,MFA的准确率随维度增加发生了微弱下降;MFA-L2/L1的准确率逐渐上升,超过MFA之后保持稳定。由相关工作中和方法介绍可知,两种方法均采用L2损失定义类内散度,两者的区别在于对类间散度定义不同。与MFA相比,MFA-L2/L1采用L1损失定义类间散度,提升了该方法的判别性。该实验结果表明,由于采用了L1损失对类间样本赋予更大的惩罚,MFA-L2/L1较MFA具有更好的性能。

\subsection{FEI数据集实验} 
由于MFA-L2/L1与LDA-NgL1同属非贪婪算法,且基于UCI数据集上都具有较好的排名与稳定性,该小节实验基于FEI \citep{FEI10}数据集对两者进行对比。该数据集同样为公开的人脸识别领域数据集,共包含200人(男性、女性各100人),每人采集了14张RGB图像,包含了不同角度、光照等14个场景变化,实验选取6个固定场景的1200张图像(均为正面人脸)进行算法比较。

由于前两小节实验已考察了不同噪声程度和噪声类型下的算法性能,本小节只考虑一种噪声类型。实验事先将1200张图像统一转为32×32像素大小的灰度图像,每个类别随机选取2张图像进行加噪。参考文献\citep{chen2018robust}的方式,随机选取图像$25\%$的像素点,将其从原像素值$a$翻转至$255-a$作像素噪声。每个类别随机选取3张干净图像用作测试样本,其余样本全用作训练,该过程重复10次。实验设置与前两小节保持一致,实验对比了投影矩阵维度在100维至250维之间的性能变化,其结果见图~\ref{fig:FEI}。

\begin{figure}[htbp]
    \centering
    \includegraphics[width=0.8\textwidth]{chap2_FEI}
    \caption{MFA-L2/L1与LDA-NgL1两个方法基于FEI数据随维度变化的准确率对比}
    \label{fig:FEI}
\end{figure}
           
观察图~\ref{fig:FEI},随维度增加,MFA-L2/L1和LDA-NgL1两个方法的性能差距逐渐变大。LDA-NgL1的性能随维度增加发生了一定的波动和下降;MFA-L2/L1在不同维度下的性能始终高于LDA-NgL1,且相对稳定。由相关工作和方法介绍可知,两种方法均采用L1损失刻画类间散度,两者区别主要在于对类内散度定义不同。与LDA-NgL1相比,MFA-L2/L1采用L2损失刻画类内散度以加大惩罚力度,所以具有更好的判别性。

\subsection{人脸数据集综合实验} 
为了更全面比较MFA-L2/L1与相关方法的性能,该小节基于5个人脸数据集作了进一步的实验对比。除了LDA-NgL1、MFA-NgL1、LDA-L1、MFA和DNS以外,对比方法加入了经典的度量学习方法KISSME \citep{Kostinger2012}。在这些方法中,DNS主要学习类内零空间作为度量矩阵;依赖于正样本对和负样本对,KISSME主要计算出两个协方差的逆矩阵之差作为度量矩阵。这两个方法的度量矩阵维数固定、无需设定。鉴于此,本小节均在固定维度下进行实验对比。实验设置仍与之前一致,对训练和测试过程重复10次,取实验平均结果作为报告。考虑上一小节实验未进行PCA处理,算法间并无较大性能差异,该节保留PCA中$95\%$贡献率,对所有数据进行预处理。5个数据集的具体信息和加噪方式具体如下:

1) Senthil数据集$\footnote{http://www.geocities.ws/senthilirtt/Senthil\%20Face\%20Database\%20Version1}$: 该数据集由5个人的80张人脸图像构成,每个类别各有16张图像,所有图像均转为32×32像素大小的灰度图像。实验从每个类别中随机选取4张图像作为噪声样本,与上一节加噪方式类似,加入$20\%$的像素噪声。为保证训练集和测试集各占一半,每个类别中随机选取6个干净样本、2个噪声样本用于训练,其余样本全用于测试。该数据集的投影矩阵输出维度设置为25。

2) Yale数据集$\footnote{http://vision.ucsd.edu/content/yale-face-database}$ \citep{yale97}: 该数据集由15个人的165张人脸图像构成,每个类别各有11张图像,所有图像均转为32×32像素大小的灰度图像。实验从每个类别中随机选取3张图像作为噪声样本,加入$15\%$面积比的高斯块噪声。为保证训练集和测试集大约各占一半,实验从每个类别中随机选取5张干净样本用于测试,其余样本全用于训练。该数据集的投影矩阵输出维度设置为40。

3) ORL数据集$\footnote{http://cam-orl.co.uk/facedatabase.html}$ \citep{ORL94}: 该数据集由40个人的400张人脸图像构成,每个类别各有10张图像,所有图像均转为33×28像素大小的灰度图像。实验从每个类别中随机选取3张图像作为噪声样本,加入$15\%$面积比的高斯块噪声。为保证训练集和测试集各占一半,每个类别随机选取5张干净图像用于测试,其余全用于训练。该数据集的投影矩阵输出维度设置为35。

4) Caltech数据集$\footnote{http://www.vision.caltech.edu/html-files/archive.html}$ 该数据集由31个人的450张人脸图像构成,其中有5个类别均只含一张图像,其余每个类别各包含5到29张图像不等。实验先将其裁剪为193×162像素大小的灰度图像,后转为32×27像素大小。考虑该数据集类别的不平衡性,实验将样本数不超过5的类别全用作噪声样本,从其余每个类别中随机选取5张图像作噪声样本,与上一小节加噪方式类似,加入$25\%$的像素噪声。实验将仅含一个样本的类别用于训练;对于其余每个类别,实验从中随机选取一半的样本用于训练,剩下一半样本用于测试。该数据集的投影矩阵输出维度设置为60。

5) UMIST数据集$\footnote{https://www.visioneng.org.uk/datasets/}$ \citep{umist98}: 该数据集由20个人的575张人脸图像构成,每个类别包含19到48张图像不等。每幅图像转为32×32像素大小的灰度图像。对每个类别随机选取10张图像作为噪声样本,与上一小节加噪方式类似,加入$20\%$的像素噪声。对每个类别随机选取大约一半样本数的干净样本作测试,其余样本均用作训练。该数据集的投影矩阵输出维度设置为25。
\begin{table}[htbp]
{\renewcommand\baselinestretch{0.83}\selectfont
\renewcommand{\arraystretch}{1.3}
    \caption{各方法基于5个人脸数据集实验的准确率结果(\%)}
    \label{tab:chap2_facedata}
\centering
\scalebox{1}{
\begin{threeparttable}
		\begin{tabular}{|l|c|c|c|c|c|}\hline 
\multirow{2}{*}{算法}&\multicolumn{5}{c|}{数据集/维度}\\  
\cline{2-6}
      &{Senthil/25}  &{Yale/40} &{ORL/35}  &{Caltech/60}  &{UMIST/25}\\  

\hline
DNS	&63.5	&77.87	&-	&83.81	&67.44\\
KISSME	&76.0	&80.27	&-	&94.51	&97.44\\
LDA-L1	&77.25	&79.07	&90.5	&91.53	&78.18\\
LDA-NgL1	&83.0	&86.67	&93.2	&96.65	&96.56\\
MFA-NgL1	&82.0	&88.67	&93.3	&96.56	&96.74\\
MFA	&83.0	&87.87	&92.4	&96.65	&97.89\\
MFA-L2/L1	&\textbf{84.0}	&\textbf{90.27}	&\textbf{94.1}	&\textbf{97.02}	&\textbf{98.11}\\
\hline
\end{tabular}
\footnotesize
  最优结果加粗表示。
\end{threeparttable}}
\par}
\end{table}

基于5个人脸数据集的实验结果见表~\ref{tab:chap2_facedata},整体对比得知,在不同数据集的不同噪声情形下,MFA-L2/L1方法均取得了最优性能。在MFA-L2/L1、LDA-NgL1、MFA-NgL1和LDA-L1四个迭代算法中,LDA-L1的性能较差。这一现象的原因在于,该方法采用逐向量方式求解的贪婪解法,所以算法精确度不佳,该现象从另一层面上也反映了非贪婪算法的优势。

比较LDA-NgL1、MFA-NgL1和MFA三个方法,发现前两者并没有太大的优势。这一现象说明,在该实验噪声环境下,采用L1损失刻画类内散度和类间散度带来的提升并不明显。DNS方法在对比中性能较差,其原因在于每个数据集的线性零空间是有限的,导致其在该情况下的表达能力有限。由于KISSME方法计算的度量矩阵与原空间空大小一致,未能有效刻画数据潜在的低维结构,限制了其相似性的刻画能力,该缺点在文献\citep{Liao2015}中有所验证。综上分析表明,MFA-L2/L1通过利用不同的距离作为损失以刻画类内散度和类间散度,使得其在该本实验中优于其它算法,具有较好的鲁棒性和判别性。

\subsection{算法运行时间分析} 
为了对算法收敛性情况进行对比,实验记录了Caltech人脸数据集上各算法的运行时间,具体报告见表格~\ref{tab:time_rmfa}。从计算方式上分析,DNS和KISSME属于闭式求解算法,无需迭代,两者运行的时间显然最短。而其余四种方法均属于迭代方法,最大外部迭代次数均为100次,所运行时间相对非迭代方法更长。其中,采用同一种模型和求解方式的LDA-NgL1和MFA-NgL1所运行时间最长,虽然LDA-NgL1属于非贪婪式方法,但其求解方式涉及矩阵的梯度迭代计算,收敛过程缓慢。由于MFA-NgL1基于相似性生成的样本对,其训练样本数量实质远大于LDA-L1和LDA-NgL1中的样本数,所以其运行时间最长。而更为有趣的是,在同样的样本对数量下,所提方法MFA-L2/L1的运行时间低于非贪婪方法LDA-L1。该现象的原因在于,MFA-L2/L1虽属于非贪婪方法,但其内部迭代只需一次即可收敛,每步迭代实质是闭式求解,无需进行LDA-NgL1的梯度计算,该方法收敛性较快。综合分析,由于其简洁的求解过程,MFA-L2/L1所运行的时间较短,这从一定程度上也反映了算法具有较快的收敛性。
\begin{table}[!htbp]
{\renewcommand\baselinestretch{0.86}\selectfont
\renewcommand{\arraystretch}{1.3}
    \caption{各方法在Caltech数据集上的运行时间对比}
    \label{tab:time_rmfa}
\centering
\scalebox{0.9}{
\begin{threeparttable}
		\begin{tabular}{|l|c|c|c|c|c|c|}\hline 
算法        &DNS    &KISSME    &LDA-L1   &LDA-NgL1   &MFA-NgL1   &MFA-L2/L1    \\ 
\hline
时间(s)   &0.31	&0.02	  &4.71	    &16.25  	&57.86      &3.68  \\
\hline
\end{tabular}
\footnotesize

\end{threeparttable}}
\par}
\end{table}

\section{本章小结}
针对特征噪声对度量学习的影响,本章提出了一种非贪婪度量学习方法MFA-L2/L1,随后推导了一种高效的求解算法。MFA-L2/L1方法采用L2损失和L1损失分别对应定义类内散度和类间散度,同时赋予两类样本更大的惩罚,以提升投影矩阵的判别性。针对目标函数非凸带来的求解困难,本章推导了一种非贪婪的迭代求解算法,并证明了算法的收敛性。在5个UCI数据集的不同噪声程度、AR和FEI数据集的不同维度、Senthil等5个人脸数据集上进行了大量仿真实验,结果分析表明MFA-L2/L1方法具有一定抗噪性和判别性。